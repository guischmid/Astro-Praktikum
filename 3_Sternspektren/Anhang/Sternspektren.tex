\documentclass[a4paper,12pt]{article}
\usepackage[utf8]{inputenc}
\usepackage[ngerman]{babel}
\usepackage{amsmath, amssymb}
\usepackage{graphicx}
\usepackage{float}
\usepackage{geometry}
\usepackage{booktabs}
\geometry{a4paper, margin=1in}

\title{Spektralklassenbestimmung und Leuchtkraftanalyse von Sternen}
\author{Astronomisches Praktikum \\
Sommersemester 2024\\\\
Guilherme Schmid}
\date{}

\begin{document}

\maketitle

\section*{Zielsetzung}
Ziel des Versuches war es, die Spektralklassen der Sterne anhand ihrer Spektren zu bestimmen und daraus ihre effektive Temperatur, absolute Helligkeit, bolometrische Helligkeit, Leuchtkraft sowie den Radius abzuleiten. Dies umfasst die Zuordnung der Sterne zu den Harvard-Spektralklassen durch Vergleich mit Normspektren und die Berechnung relevanter astrophysikalischer Parameter.

\section*{Durchführung}
1. \textbf{Spektralklassenbestimmung}: 
   Die Spektren unbekannter Sterne wurden mit Normspektren (Abb. 3.1) verglichen, um ihre Spektralklasse zu bestimmen. Die Einordnung erfolgte durch Beobachtung dominanter Spektrallinien bei verschiedenen Temperaturen. Eine genauere Zuordnung wurde durch Zehnteleinteilung der Spektralklassen erreicht.

2. \textbf{Berechnung der Sternparameter}:
   \begin{itemize}
       \item \textbf{Effektive Temperatur (\( T_{\text{eff}} \))}: Bestimmung der Temperatur durch Interpolation der Normspektren.
       \item \textbf{Entfernung (\( d \))}: Berechnung der Entfernung in Parsec (pc) basierend auf der jährlichen Parallaxe (\( \pi \)) gemäß Gleichung (3.2):
       \[
       d = \frac{1}{\pi}
       \]
       \item \textbf{Absolute Helligkeit (\( M_V \))}: Berechnung aus der scheinbaren Helligkeit (\( m_V \)) und der Entfernung (\( d \)) mittels des Entfernungsmoduls (Gleichung 3.6):
       \[
       M_V = m_V + 5 - 5 \log d
       \]
       \item \textbf{Bolometrische Helligkeit (\( M_{\text{bol}} \))}: Korrektur der visuellen Helligkeit durch Hinzufügen des Bolometrischen Korrekturwertes (BC) abhängig von der Spektralklasse:
       \[
       M_{\text{bol}} = M_V + BC
       \]
       \item \textbf{Leuchtkraft (\( L_* \))}: Berechnung der Leuchtkraft relativ zur Sonne aus der bolometrischen Helligkeit (Gleichung 3.8):
       \[
       \log \frac{L_*}{L_\odot} = -0,4 (M_{\text{bol}} - M_{\text{bol},\odot})
       \]
       \item \textbf{Radius (\( R_* \))}: Berechnung des Radius des Sterns aus der Leuchtkraft und der effektiven Temperatur:
       \[
       \frac{R_*}{R_\odot} = \sqrt{\frac{L_*}{L_\odot}} \left(\frac{T_\odot}{T_{\text{eff}}}\right)^2
       \]
   \end{itemize}

\section*{Auswertung}
Die folgenden Tabellen zeigen die berechneten Werte für die Entfernungen, absoluten Helligkeiten, bolometrischen Helligkeiten, Leuchtkräfte und Radien der Sterne.

\begin{table}[H]
    \centering
    \begin{tabular}{cccccc}
        \toprule
        Nr. & \( d \) (pc) & \( M_V \) & \( M_{\text{bol}} \) & \( L_* / L_\odot \) & \( R_* / R_\odot \) \\
        \midrule
        1  & 5.49  & 4.90 & 4.78 & 0.96  & 0.98 \\
        2  & 35.71 & 2.64 & 2.52 & 7.76  & 2.79 \\
        3  & 8.13  & 0.55 & 0.43 & 52.99 & 7.28 \\
        4  & 3.42  & 7.53 & 7.41 & 0.09  & 0.29 \\
        5  & 9.26  & 3.67 & 3.55 & 3.00  & 1.73 \\
        6  & 17.86 & 0.84 & 0.72 & 40.52 & 6.37 \\
        7  & 71.43 & -1.47 & -1.59 & 340.21 & 18.44 \\
        8  & 12.99 & 3.63 & 3.51 & 3.10  & 1.76 \\
        9  & 5.59  & 5.96 & 5.84 & 0.36  & 0.60 \\
        10 & 66.67 & -0.32 & -0.44 & 117.98 & 10.86 \\
        11 & 3.30  & 6.21 & 6.09 & 0.29  & 0.54 \\
        12 & 9.09  & 5.71 & 5.59 & 0.46  & 0.68 \\
        13 & 76.92 & -0.43 & -0.55 & 130.65 & 11.43 \\
        14 & 47.62 & -0.49 & -0.61 & 137.90 & 11.74 \\
        \bottomrule
    \end{tabular}
    \caption{Berechnete astrophysikalische Parameter der untersuchten Sterne.}
    \label{tab:results}
\end{table}

\section*{Fazit}
Die Ergebnisse zeigen eine breite Vielfalt in den Eigenschaften der untersuchten Sterne, von sonnenähnlichen Sternen bis hin zu sehr leuchtkräftigen Riesensternen. Die Variation der Leuchtkraft und der Radien spiegelt die Unterschiede in der Masse und dem Entwicklungsstadium der Sterne wider.
\section*{Anhang}
Die Berechnungen wurden mit Hilfe eines Python-Skripts durchgeführt, welches die erforderlichen Messdaten und Berechnungen automatisiert hat.


\end{document}
